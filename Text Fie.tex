\documentclass[12pt]{article}


\usepackage[utf8]{inputenc}
\usepackage{mathpazo}
\usepackage[T1]{fontenc}

\usepackage{graphicx}
\graphicspath{{images/}}
\usepackage[labelfont=bf]{caption}
\begin{document}
\begin{titlepage}
\newcommand{\HRule}{\rule{\linewidth}{0.5mm}}
\center
\textsc{\LARGE National Institute of Technology Raipur}\\[1.0cm]
\begin{figure}[h]
\centering
\includegraphics[scale=0.3]{nit.jpg}
\end{figure}
\textsc{\Large Biomedical Engineering}\\[0.5cm] 
\textsc{\large Assignment}\\[0.5cm]
\HRule\\[0.4cm]
	
	{\huge\bfseries Evolution of Modern Health Care System}\\[0.1cm]
	\HRule\\[1.1cm]
	\begin{minipage}{0.4\textwidth}
		\begin{flushleft}
			\large
			\textit{Submitted By:}\\
          		Name : Shivam Gupta \\
			   Roll No. : 21111055\\
			   Semester : First\\
			   Branch - Biomedical Engineering 
			   
			\end{flushleft}
	\end{minipage}
	~
	\begin{minipage}{0.5\textwidth}
		\begin{flushright}
			\large
			\textit{Under The Supervision Of:}\\
			Dr. Saurabh Gupta\\
			Department Of Biomedical Engineering\\
			NIT Raipur
		\end{flushright}
	\end{minipage}
	\vfill\vfill\vfill 
	
	
	\vfill 
\end{titlepage}

\clearpage
\newpage


\textsc{\Large Evolution of Modern Health Care System}\\[1.5cm]

\section*{Traditional Server-Based EMR systems }
New computer technology such as desktop computers to enter, and local servers to store data laid the foundation for the first Electronic Medical Records (EMR) system, developed in the early 1970s. EMR technology was one of the essential starting points for the future development of RTHS’s within the healthcare industry. EMRs emerged as a solution to eliminate the errors and inefficiencies resulting from paper charting. The traditional EMR system is a digital version of a patient’s charts. Patient data is stored on an internal server, containing patient medical history, diagnoses, medications, consults, orders, laboratory test results, and more. Although patient information is consolidated in a central location, there is still a delay in accessing and retrieving relevant patient information creating a new set of problems. For example, a clinician must log into the system, search for the right patient, and navigate through a myriad of screens before locating the relevant information. 

 

\section*{Web-Based EMR Systems }
Web and browser technology emerged in the 1990s, allowing faster and easier access to information online, paving the way for online health information, and eventually to the development of web-based EMRs. There are several key differences between a web-based EMR and a traditional server-based EMR. As mentioned, traditional EMR systems store patient data on local, internal servers. Web-based EMR’s are cloud-based and store data on external servers that can be accessed from any device with an internet connection that has the required security access. Web-based EMR’s offer many advantages to physicians. Due to being hosted on the cloud, users are not impacted by system failures due to power outages, or hardware and software issues.




\section*{Digital Twins/Real-Time Health System}
Fast-forward to the Smart Hospital Digital Twin - a technology that achieves a seamless interconnected real-time health system. Digital Twins have evolved rapidly over the past few years. Smart Hospital Digital Twins connect previously disparate systems to give clinicians and patients an overview of the bigger picture. Unlike other solutions, Digital Twins present a holistic approach, looking at the entire hospital environment combining data from various subsystems to provide new insights in real-time, based on the interactions between people, processes, and connected things
\linebreak
 
Digital Twins enable healthcare providers to continue providing high-quality care while unlocking the following use cases to protect all staff and patients from spread of the virus: 
\begin{itemize}
\item Contact Tracking: Through the Digital Twin empowered tenant experience app, infection contact tracing identifies the location of devices within a space and tracks the patient-clinician interactions. This allows hospital staff to keep a record of contact events and implement infection precautions around individuals who require it

\item Managing Patient Flow: Managing patient flow throughout the hospital is important to track those who may have had contact with the virus. This may include workflows such as ensuring the correct procedures are carried out by notifying porters and other staff of infection precautions of patients. Also, the Digital Twin empowered tenant experience app assists staff and visitors find their way. The app directs them where to park and which entrance to come through according to space density. 





\item  Bed Resource Planning: As COVID-19 has led to increases in ICU admissions, the Digital Twin’s ability to augment planning of bed resources is vital. Hospital staff now more easily plan how many beds are needed, as the solution identifies when patients from surrounding healthcare facilities will be arriving. By easily keeping track of patient flow and available beds, hospital staff can effectively keep beds in a unit spaced out and keep beds available for infected patients. 
\end{itemize}





















\end{document}
