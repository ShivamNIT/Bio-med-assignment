\documentclass[12pt]{article}

\usepackage{geometry}
 \geometry{
 a4paper,
 total={180mm,260mm},
 left=15mm,
 right=15mm,
 top=16mm,
 bottom=16mm,
 }


\usepackage[utf8]{inputenc}
\usepackage{mathpazo}
\usepackage[T1]{fontenc}

\usepackage{graphicx}
\graphicspath{{images/}}
\usepackage[labelfont=bf]{caption}
\begin{document}
\begin{titlepage}
\newcommand{\HRule}{\rule{\linewidth}{0.5mm}}
\center
\textsc{ \huge National Institute of Technology \\ \vspace{3mm} Raipur}\\[0.8cm]
\begin{figure}[h]
\centering
\includegraphics[scale=1]{nit.jpg}
\end{figure}

\centering


%\textsc{\Large Biomedical Engineering}\\[0.5cm] 
\vspace{5mm}
\textsc{\huge Term Paper}\\[0.5cm]
\vspace{5mm}
	\HRule\\[0.4cm]
	{\huge\bfseries Computed Tomography(CT Scan) }\\[0.1cm]
	\HRule\\[1.5cm]
	

	\begin{minipage}{0.4\textwidth}
		\begin{flushleft}
			\Large
			\textit{Submitted By:}\\
          		Name : Shivam Gupta \\
			   Roll No. : 21111055\\
			   Semester : First\\
			   Branch - Biomedical Engineering 
			   
			\end{flushleft}
	\end{minipage}
	~
	\begin{minipage}{0.5\textwidth}
		\begin{flushright}
			\Large
			\textit{Under The Supervision Of:}\\
			Dr. Saurabh Gupta\\
			Department Of Biomedical Engineering\\
			NIT Raipur
		\end{flushright}
	\end{minipage}
	\vfill\vfill\vfill 
	
	
	\vfill 
\end{titlepage}



\clearpage
\newpage

\textsc{\huge Computed Tomography Scanner }\\[0.3cm]

\section*{ \huge Abstract}
\large
Computed tomography (CT) has changed diagnostic decision-making since its inception in the 1970s. The increased radiation exposure received by patients is one of the key concerns linked with the growing use of CT. The association between ionising radiation and the development of neoplasia has mostly been based on extrapolating data from studies of survivors of the 1945 atomic bombings in Japan, as well as estimations of the higher relative risk of neoplasia in people working in the nuclear sector. The link between low-dose radiation from diagnostic imaging exams and oncogenesis, on the other hand, is still unknown. Significant progress has already been made in terms of radiation dose reduction thanks to improved technologies.
\\
Several dosage optimization measures are easily available, including eliminating unneeded images at the ends of collected series, reducing the number of phases captured, and using automatic exposure management rather than fixed tube current procedures. Furthermore, in recent years, new picture reconstruction algorithms that lower radiation dose have been developed with promising results. These methods make use of iterative reconstruction algorithms to provide diagnostic-quality images with less image noise while using fewer radiation doses.
\begin{figure}[h]
\centering
\includegraphics[scale=0.4]{ctscan.jpg}
\end{figure}



\section*{ \huge Introduction}
\large
CT scans are commonly used in radiology to provide high-quality diagnostic images, however they come with a significant dose of medical radiation. The effective dose for most CT exams is between 1 and 10 millisieverts, although it can be greatly raised, up to three times, in instances where more scans are necessary. Due to the high radiation doses involved with CT exams, it's important to keep track of patient dose data and CT protocol scanning methods, as well as to use system optimization and dose reduction approaches.

It's worth noting that when organ-specific cancer risk is adjusted for present levels of CT use, it's possible that 1.5-2 percent of cancers are caused by the ionising radiation used in CT , however there's a lot of controversy about these assumptions. It's worth noting that the effective dose of 10 mSv is linked to a higher risk of cancer death.


\section*{ \huge How does CT work}
A CT scanner, unlike a traditional x-ray, uses a motorised x-ray source that rotates around the circular entrance of a donut-shaped frame known as a gantry. A CT scan involves the patient lying on a bed that moves slowly through the gantry as an x-ray tube rotates around them, shooting narrow beams of x-rays through the body. CT scanners use digital x-ray detectors instead of film, which are placed directly opposite the x-ray source. The detectors take up the x-rays as they leave the patient and send them to a computer.
\\
\\
The CT computer employs advanced mathematical procedures to produce a 2D image slice of the patient every time the x-ray source completes one full revolution. The thickness of the tissue shown in each imaging slice varies each CT equipment, however it commonly falls between 1 and 10 millimetres. The image is saved once a whole slice is completed, and the motorised bed is slowly dragged forward into the gantry. A new image slice is created by repeating the x-ray scanning process. This procedure is repeated until the necessary number of slices has been gathered.
\\ 
\\
\begin{figure}[h]
\centering
\includegraphics[scale=0.4]{s3.png}
\end{figure}
\\
The skeleton, organs, and tissues, as well as any abnormalities the physician is seeking to identify, can either be presented individually or piled together by the computer to form a 3D representation of the patient. This method provides a number of benefits, including the ability to rotate the 3D image in space or see slices in order, making it easier to pinpoint the specific location of a problem.

\section*{ \huge When would we get a CT scan}

CT scans can be used to detect disease or injury in a variety of bodily areas. CT scans, for example, have proven an effective screening technique for detecting potential cancers or lesions in the abdomen. When various sorts of cardiac illness or anomalies are suspected, a CT scan of the heart may be recommended. CT scans of the head can be used to detect injuries, tumours, clots that cause strokes, haemorrhage, and other diseases. It can detect malignancies, pulmonary embolisms (blood clots), excess fluid, and other diseases like emphysema or pneumonia by imaging the lungs. When imaging complex bone fractures, badly degraded joints, or bone malignancies, a CT scan is very beneficial.




\section*{ \huge CT Scan with Contrast}

Dense substances, such as bones, are easily seen on a CT scan. Soft tissues, on the other hand, do not show up as well. In the photograph, they may appear to be dim. A special dye known as a contrast substance may be required to help them appear more early. They look white on the scan and block X-rays, highlighting blood arteries, organs, and other structures. 
\\
\\
Iodine or barium sulphate are the most common contrast materials. These medications could be delivered to us in one of three ways:

\begin{itemize}


\item Injection: The medications are injected into a vein directly. This is done to make our blood arteries, urinary tract, liver, and gallbladder more visible in the photograph.

\item Orally:Drinking a drink containing contrast material can help improve scans of our digestive tract, which is the conduit through which food passes through our bodies.

\item Enema:The contrast material can be injected into our rectum if our intestines are being scanned.

\end{itemize}

\begin{figure}[h]
\centering
\includegraphics[scale=0.8]{process.jpg}
\end{figure}


\section*{ \huge Uses}
\large
\begin{itemize}


\item Many cancers, such as liver, lung, and pancreatic tumours, are routinely diagnosed using CT scans.

\item A CT scan of the abdomen can identify a tumour as well as any swelling or inflammation in the internal organs surrounding. It can reveal any spleen, kidney, or liver lacerations.


\item A CT scan is useful for planning areas for radiotherapy and biopsies because it detects abnormal tissue, and it can also provide significant data on blood flow and other vascular issues.

\item It can aid a physician in determining bone disorders, bone density, and the condition of a patient's spine.

\item It can also provide important information regarding a patient's injuries to their hands, feet, and other skeletal structures. Even little bones, as well as their surrounding tissue, are clearly apparent.



\end{itemize}



\section*{ \huge Risk}
\large
\begin{itemize}
\item Radiation can cause birth abnormalities if it is received during pregnancy.

\item There is a possibility of an allergic reaction to contrast dye if it is utilised. Patients with allergies or sensitivities to medicines, contrast dye, iodine, or shellfish should inform their doctor.


\item Patients who have renal failure or other kidney issues should inform their doctor. The contrast dye might cause kidney failure in some people, especially if they are using Glucophage (a diabetic drug).


\end{itemize}




\section*{ \huge Conclusion}
\large
CT scan has made a major impact on the practice of dentistry, particularly in oral and maxillofacial diagnosis, surgery and management of a wide variety of oral lesions. Advances in computer softwares already allow 3D visualization of anatomy and pathology, but further improvement in clinical performance is expected.

\vspace{10mm}


\section*{ \huge References }
\large
\vspace{3mm}
\begin{itemize}

\item https://www.nibib.nih.gov/science-education/science-topics/  computed-tomography-ct 


\item  https://www.medicalnewstoday.com/articles/153201finding-a-radiologist/

\item https://www.verywellhealth.com/what-is-a-cat-scan-189603

\end{itemize}

\enddocument
