

\documentclass[11pt]{article}

\usepackage[utf8]{inputenc} % Required for inputting international characters
\usepackage[T1]{fontenc} % Output font encoding for international characters

\usepackage{mathpazo} % Palatino font



\date{}
\begin{document}
\date{}


\begin{titlepage} 
	\newcommand{\HRule}{\rule{\linewidth}{0.5mm}} % Defines a new command for horizontal lines, change thickness here
	
	\center % Centre everything on the page
	
	
	
	\textsc{\LARGE NATIONAL INSTITUTE OF TECHNOLOGY   RAIPUR}\\[1.5cm] % Main heading such as the name of your university/college
	
	\textsc{\Large BIOMEDICAL ENGINEERING }\\[0.5cm] % Major heading such as course name
	
	\textsc{\large ASSIGMENT}\\[0.5cm] % Minor heading such as course title
	
	
	
	
	
	%------------------------------------------------
	%	Title
	%------------------------------------------------
	
	\HRule\\[0.4cm]
	
	{\huge\bfseries Medical Devices}\\[0.4cm] % Title of your document
	
	\HRule\\[1.5cm]
	
	%------------------------------------------------
	%	Submitted by;(s)
	%------------------------------------------------
	
	\begin{minipage}{0.4\textwidth}
		\begin{flushleft}
			\large
			\textit{submitted by}\\
			Name- \textsc{Shivam Gupta} % 
			Roll no.-21111055 %
			
		E-Mail Id-ssgg9302@gmail.com %
			
			
		\end{flushleft}
	\end{minipage}
	~
	\begin{minipage}{0.4\textwidth}
		\begin{flushright}
			\large
			\textit{Under the supervision of}\\
			Dr. Saurab \textsc{Gupta} % Supervisor's name
		\end{flushright}
	\end{minipage}
	
	% If you don't want a supervisor, uncomment the two lines below and comment the code above
	%{\large\textit{Author}}\\
	%John \textsc{Smith} % Your name
	
	%------------------------------------------------
	%	
	%------------------------------------------------
	
	\vfill\vfill\vfill % Position the date 3/4 down the remaining page
	
	{\large\today} % Date, change the \today to a set date if you want to be precise
	
	%------------------------------------------------
	%	Logo
	%------------------------------------------------
	
	%\vfill\vfill
	%\includegraphics[width=0.2\textwidth]{placeholder.jpg}\\[1cm] % Include a department/university logo - this will require the graphicx package
	 
	%----------------------------------------------------------------------------------------
	
	\vfill % Push the date up 1/4 of the remaining page
	\date{}
\end{titlepage}
\clearpage
\newpage




\textsc{\LARGE OXIMETER}\\[1.5cm]
\section{What Is a Pulse Oximeter}
A pulse oximeter is a compact, portable device that measures the quantity of oxygen carried throughout the body. This painless device attaches to your fingertip and sends two wavelengths of light through it to monitor your pulse rate and oxygen levels in your system. The oximeter's screen will reveal the percent of oxygen in your blood that comes from your heart, as well as your current pulse rate, once it has completed its examination.
\subsection{Medical uses}
A pulse oximeter is a medical device that produces a photoplethysmogram, which can be subsequently processed into various data, by indirectly monitoring the oxygen saturation of a patient's blood (rather than directly measuring oxygen saturation through a blood sample). [4] A multiparameter patient monitor could include the pulse oximeter. The pulse rate is usually displayed on most monitors. Portable, battery-operated pulse oximeters are also available for use on the go or at home to monitor blood oxygen levels.
\section{Mechanism}
The proportion of blood filled with oxygen is displayed on a blood-oxygen monitor. It determines what percentage of haemoglobin, the oxygen-carrying protein in the blood, is loaded. For patients without pulmonary disease, normal SaO2 ranges from 95 to 99 percent. [requires citation] The "saturation of peripheral oxygen" (SpO2) value on a blood-oxygen monitor can be used to determine arterial pO2 for a person breathing room air at or near sea level.


\subsection{Blood Oxygen Saturation (SpO2)}
SpO2 reading is an estimation of the amount of oxygen in your blood. An SpO2 reading of 95 greater is generally considered to be a normal oxygen level. However, an SpO2 reading of 92 or less (at sea level) suggests that your blood is poorly saturated. Insufficient saturation can cause a range of adverse health conditions—including chest pain, shortness of breath and increased heart rate.
\section*{Advantages}
Pulse oximeters are useful for people who have conditions that affect oxygen saturation. For example, a sleep specialist might recommend a pulse oximeter to monitor the nighttime oxygen saturation level of someone with suspected sleep apnea or severe snoring.

Pulse oximetry can also provide feedback about the effectiveness of breathing interventions, such as oxygen therapy and ventilators.

Some doctors use pulse oximetry to assess the safety of physical activity in people with cardiovascular or respiratory problems, or may recommend that a person wears a pulse oximeter while exercising. A doctor may also use pulse oximetry as part of a stress test.


\section*{Limitation}
Pulse oximetry is a noninvasive, accurate, and safe method for the measurement of oxygen saturation during intravenous sedation or general anesthesia. Several factors should be considered with its use, since these variables will either alter the accuracy of the readings or may cause harm to the patient. These factors include changes in the strength of the arterial pulse, body movements, dyshemoglobinemias, plasma lipids and bilirubin, color interferences, venous pulsations, and several physical factors. Awareness of these variations will help the clinician become more knowledgeable in the use of the pulse oximeter//

\clearpage
\newpage
\textsc{\LARGE ECG Scanner}\\[1.5cm]
\section*{What is ECG}
ECG, also referred to as EKG, is the abbreviation of the word electrocardiogram – a heart test that tracks the electrical activity of your heart and records it on a moving paper or shows it as a moving line on a screen. An ECG scan is used to analyze the heart’s rhythm and detect irregularities and other cardiac issues that might lead to serious health problems such as a stroke or heart attack.

\section*{Mechanism}
To get an ECG trace, an ECG monitor is needed to record it. As the electrical signals move through the heart, the ECG monitor records the strength and the timing of these signals in a graph called a P wave. Traditional monitors use patches and wires to attach electrodes to the body and communicate the ECG trace to a receiver.

\section*{Purpose of an ECG Test}
An ECG detects your heart's electrical rhythm and produces what's known as a tracing, which looks like squiggly lines. This tracing consists of representations of several waves that recur with each heartbeat, about 60 to 100 times per minute. The wave pattern should have a consistent shape. If your waves are not consistent, or if they do not appear as standard waves, this is indicative of heart disease.

There are a variety of characteristic changes that occur with different heart problems, and your healthcare provider can look at your ECG wave patterns to see if they are suggestive of certain types of heart disease
\subsection*{Condition}
There are a number of conditions that can be detected when your healthcare provider checks your pulse, such as tachycardia (rapid heart rate), bradycardia (slow heart rate), and arrhythmia (irregular heart rate). EKG wave patterns can verify these alterations in your heart rhythm, and certain changes in the shape of the waves provides information about the specific type of heart disease and which region in the heart is affected.

\section*{Advantages}
\begin{itemize}
\item ECG is helpful to measure three basic parameters of clinical interest viz. rhythm and heart rate, axis of the heart and state of myocardial muscle

\item ECG is used to detect the cardiac conditions of the patients after surgical or any other operation and after application of anesthesia.

\item ECG test is quick, painless and safe.

\item ECG test is cheap in cost.
\end{itemize}

\section*{Limitation}
\begin{itemize}
\item The ECG reveals the heart rate and rhythm only during the few seconds it takes to record the tracing. If an arrhythmia (heart rhythm irregularity) occurs only intermittently, an ECG might not pick it up, and ambulatory monitoring may be required
\item The ECG is often normal or nearly normal with many types of heart disease, such as coronary artery disease.
\item Sometimes, abnormalities that appear on the ECG turn out to have no medical significance after a thorough evaluation is done
\end{itemize}

\clearpage
\newpage



\textsc{\LARGE  Enhanced External Counter Pulsation}\\[1.5cm]
\section*{What is EECP}

EECP is a mechanical procedure in which long inflatable cuffs (like blood pressure cuffs) are wrapped around both of the patient’s legs. While the patient lies on a bed, the leg cuffs are inflated and deflated synchronously with each heartbeat.
The inflation and deflation are controlled by a computer, which uses the patient’s electrocardiogram (ECG) to trigger inflation early in diastole (when the heart relaxes and is filled with blood), and deflation just as systole (heart contraction) begins.

\section*{Mechanism}
The mechanism for the apparent sustained benefits seen with EECP is unknown. Some evidence suggests EECP can help induce the formation of collateral vessels in the coronary artery tree by stimulating the release of nitric oxide and other growth factors within the coronary arteries.
There's also evidence that EECP may act as a form of “passive” exercise, leading to the same sorts of persistent beneficial changes in the autonomic nervous system that are seen with real exercise
\section*{Medical uses}
The FDA approved the CardiAssistTM ECP system for the treatment of angina, acute myocardial infarction and cardiogenic shock under a 510(k) submission in 1980 Since then, additional ECP devices have been cleared by the FDA for use in treating stable or unstable angina pectoris, acute myocardial infarction, cardiogenic shock, and congestive heart failure.[2]

Studies have found EECP to be beneficial for patients with erectile dysfunction and some COPD patients. Additionally, improvements in exercise endurance in the non-diseased patient has been found in research studies.
\section*{Advantage}
\begin{itemize}
\item There is no hospitalization, as the procedure is an outpatient procedure
\item There is no risk of infection, as the procedure is non-invasive
\item There is no bleeding or risk of blood clots.
\item There is no downtime. It is simply a walk-in and walk-out procedure/session every time
\item  There is no risk of complications in EECP, as it is an external procedure performed by mechanical means

\end{itemize}
\section*{CONCLUSION}
EECP has been used in the treatment of angina for the past two decades with a record of safety and, more recently, several publications which support its efficacy. It is approved by the FDA for the treatment of chronic or unstable angina and in patients with congestive heart failure. Treatment has been associated with improved exercise tolerance and myocardial perfusion, as evidenced by nuclear imaging and positron emission tomography. More research will hopefully shed additional light on the mechanism of action and verify the longterm attenuation of symptoms in patients with unstable angina pectoris and in those with congestive heart failure.


\clearpage
\newpage

\textsc{\LARGE Audiometer}\\[1.5cm]
\section*{What is Audiometer}
An audiometer is a subjective device that is used to evaluate the hearing threshold of a person. An audiologist or other trained personnel uses an audiometer together with special audiometric testing techniques to determine the hearing threshold and to identify as well as quantify the degree of hearing loss of a person. From these results the appropriate referral for medical treatment or possible hearing aid fitting can be prescribed. With the audiometer tones of different frequencies are presented at different levels to the patient’s ear. The hearing threshold is that level at which the person can just hear the stimulus. An audiometer is a subjective testing device, a response from the patient is needed.


\section*{Functionality}
An audiometer typically transmits recorded sounds such as pure tones or speech to the headphones of the test subject at varying frequencies and intensities, and records the subject's responses to produce an audiogram of threshold sensitivity, or speech understanding profile.

\section*{Types}
The most common type of audiometer generates pure tones, or transmits parts of speech. Another kind of audiometer is the Bekesy audiometer, in which the subject follows a tone of increasing and decreasing amplitude as the tone is swept through the frequency range by depressing a button when the tone is heard and releasing it when it cannot be heard, crossing back and forth over the threshold of hearing. Bekesy audiometry typically yields lower thresholds and standard deviations than pure tone audiometry.
\section*{Advantages}

\begin{itemize}
\item This test helps in measuring the softest or least audible sound which a person can hear.
\item It provides ear specific thresholds and uses specific tones to provide place specific responses. It helps to identify the configuration of hearing loss
\item The Pure tone based audiometer uses both air and bone conduction audiometry. Hence type of loss can also be identified via the air-bone gap.


\end{itemize}

\section*{Limitation}
\begin{itemize}
\item Audiograms obtained are very often inaccurate.
\item It is a subjective and time consuming test.
\item It does not assess or evaluate all the features of hearing
\item It does not identify the nature of the pathology.
\item The pure tone audiometry technique is not perfect to identify all the losses such as "dead regions" of the cochlea (portion of inner ear) and neuropathies.


\section*{Conclusion}
Portable audiometers have historically met a need for the provision of healthcare outside traditional settings. However, this often required the sacrifice of useful features and came in a less than ideal form factor. With the advent of mobile internet-enabled devices, it is now possible to have a full diagnostic audiometer in a long-lasting, battery-operated device. The interactive nature of the touch interface has made these devices simple to use by a wide variety of healthcare workers, in a multitude of settings. The barrier to deploy, manage, and report on hearing testing in the field has been dramatically reduced and now enables the provision of hearing healthcare to a larger segment of the population.


\end{itemize}


\clearpage
\newpage
\textsc{\LARGE Flurometer}\\[1.5cm]

\section*{What is Fluorometer}
A fluorometer or fluorimeter is a device used to measure parameters of visible spectrum fluorescence: its intensity and wavelength distribution of emission spectrum after excitation by a certain spectrum of light.[1] These parameters are used to identify the presence and the amount of specific molecules in a medium. Modern fluorometers are capable of detecting fluorescent molecule concentrations as low as 1 part per trillion.

\section*{Mechanism}
luorometers quantify biological analytes as a function of fluorescence. This requires the sample to be bound to a specific fluorescent agent and loaded into the instrument in a polypropylene tube. An extensive range of nucleic acid, cell function dyes, and fluorescent proteins are commercially available worldwide. However, fluorometers are typically limited to specific assay agents that are responsive to a single excitation channel such as an ultraviolet or blue beam.

\section*{Fluorometer Types}
There are two basic types of fluorometers: the filter fluorometers and spectrofluorometer. The difference between them is the way they select the wavelengths of incident light; filter fluorometers use filters while spectrofluorometers use grating monochromators. Filter fluorometers are often purchased or built at a lower cost but are less sensitive and have less resolution than spectrofluorometers. 
\begin{itemize}
\item Filter fluorometer
\item Spectrofluorometer
\item Integrated fluorometer
\end{itemize}

\section*{Principle of Fluorimetry}
\begin{itemize}
\item After absorption, the excitation molecules can end up in any one of the vibrational levels in the first excited electronic state
\item The light from the mercury vapor lamp is allowed to pass through the condensing lens followed by its passage through a primary filter
\item  The second possibility is that the molecules in the excited singlet state may emit an ultraviolet or visible light photon. This process is known as fluorescence
\item After absorption, the excitation molecules can end up in any one of the vibrational levels in the first excited electronic state.


\end{itemize}

\section*{Application}
Fluorimeter cuvette may be used :
\begin{itemize}
\item To perform a fluorimetric assay of the hepatic microsomal O-dealkylation reaction, using ethoxyresorufin (7-ethoxyphenoxazone) as a model substrate for the assay.
\item In the cationic liposome-mediated gene transfer method, done in order to deduce the role of helper lipids.
\item In the fluorescence experiment modified in the Silber-Porter method, for estimation of corticosterone in rat plasma
\item In fluorimeteric assay for membrane potential to determine membrane depolarization in study done to elucidate the role of membrane potential in the mechanism of action of daptomycin
\end{itemize}



















\end{document}
