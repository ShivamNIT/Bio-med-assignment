\documentclass[12pt]{article}


\usepackage[utf8]{inputenc}
\usepackage{mathpazo}
\usepackage[T1]{fontenc}

\usepackage{graphicx}
\graphicspath{{images/}}
\usepackage[labelfont=bf]{caption}
\begin{document}
\begin{titlepage}
\newcommand{\HRule}{\rule{\linewidth}{0.5mm}}
\center
\textsc{\LARGE National Institute of Technology Raipur}\\[1.0cm]
\begin{figure}[h]
\centering
\end{figure}
\textsc{\Large Biomedical Engineering}\\[0.5cm] 
\textsc{\large Assignment}\\[0.5cm]
\HRule\\[0.4cm]
	
	{\huge\bfseries Solutions to Covid19 provided by Biomedical Engineers }\\[0.1cm]
	\HRule\\[1.1cm]
	\begin{minipage}{0.3\textwidth}
		\begin{flushleft}
			\large
			\textit{Submitted By:}\\
          		Name : Shivam Gupta \\
			   Roll No. : 21111055\\
			   Semester : First\\
			   Branch - Biomedical Engineering 
			   
			\end{flushleft}
	\end{minipage}
	~
	\begin{minipage}{0.5\textwidth}
		\begin{flushright}
			\large
			\textit{Under The Supervision Of:}\\
			Dr. Saurabh Gupta\\
			Department Of Biomedical Engineering\\
			NIT Raipur
		\end{flushright}
	\end{minipage}
	\vfill\vfill\vfill 
	
	
	\vfill 
\end{titlepage}


\clearpage
\newpage





\textsc{\Large Solutions to Covid19 provided by Biomedical Engineers}\\[0.1cm]


\section*{The Role of Biomedical Engineers in COVID-19 Treatments}

Advancements in medical technology have improved the quality of patient care for coronavirus patients. To use these technologies effectively, hospitals and treatment facilities need qualified biomedical engineers. These devices have a direct impact on the survival rate of critical coronavirus cases. Any miscalibration while using such equipment can hamper the health of patients and even lead to fatalities. Hence, the need for biomedical engineers is certainly critical and evident at this juncture of the global health crisis.
\\\\
In the past two months, thousands of biomedical engineers working in the fields of medical research, device manufacturing, and drug development have come to the rescue. Indian biomedical engineers have actively responded to this crisis by adapting their knowledge, skills, and equipment for the treatment of COVID-19 patients.





\section*{Treatment mechanisms}

\subsection{Oxygen}

The first form for mild respiratory insufficiency is usually the supply of extra oxygen through a nasal cannula or a more intrusive face mask. Most of the time, the oxygen comes in the form of cylinders, either small for portability or large for stationary patients and longer-term supply.
\\
\\
Variants of oxygen supply include high flow nasal oxygen (HFNO) which delivers warmed and humidified oxygen, to avoid the drying of airways, at high flow rates - typically tens of litres/min) at body temperature and up to 100 RH and 100 oxygen.







\subsection{Patient monitoring}


An essential element of the ICU equipment is the monitoring equipment that keeps track of some of the patient vitals especially when they are ventilated and sedated but also during their recovery phase to ensure the regime of ventilation is optimised for their condition. Ventilators already provide their set of patient parameters, but usually patient monitors are separate devices as they continue to be useful after the patient can resume breathing on their own unassisted.





\subsection{Ventilators}

Patients who cannot breathe spontaneously need to be put on a ventilator. Ventilators are capable of replacing the breath function and patients in an advanced state of respiratory distress are usually intubated and sedated at the beginning of the treatment.
\\
\\
Modern ventilators are typically closed loop pressure controlled and capable of detecting spontaneous breathing to synchronise assistance for recovering patients. They also enable the control of the composition of the gas the patient breathes from normal air to 100% oxygen, usually taking their supply from the hospital’s gas supply network but can also be coupled to oxygen tanks or oxygen concentrators if used in a setting where there is no gas network






\subsection{Continuous Positive Airway Pressure (CPAP)}

The next step up in treating COVID-19 patients is Continuous Positive Airway Pressure (CPAP) which is initially intended to prevent airways collapse in sleep apnoea patients, but has been shown to be beneficial to COVID patients if applied early enough in the progression of the disease.
\\
\\
A well-fitted face mask is an essential component of a CPAP system and as such it is quite intrusive. CPAP is only appropriate for patients who are capable of some breathing strength as CPAP effectively opposes some resistance to expiration. Variants exist that either automatically adjust the level of pressure to the patients breathing characteristics (APAP) or have different levels of pressure for inspiration and expiration (BiPAP). CPAP usually supplies (filtered) air to the patient but most masks have a port for supplementing the supply with oxygen.


















\end{document}